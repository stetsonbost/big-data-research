%% Use the hmcposter class with the thesis document-class option.
\documentclass[thesis]{hmcposter}
\usepackage{graphicx}
\usepackage{natbib}
\usepackage{booktabs}
\usepackage{subfig}
\usepackage{amsmath}
\usepackage{textcomp}
\usepackage{url}
\usepackage{enumitem}
\setlist{leftmargin=2cm, labelsep=0.7cm}

%% Author of the thesis.
\author{Stetson Bost}

%% The year of your thesis poster's creation.
\posteryear{2017}

%% Thesis Title.
\title{Adaptive Nested Algorithms\\for Balanced Scheduling}

%% The name of the class for which the poster was created.
%% Generally we see posters for thesis and Clinic, but sometimes
%% other classes require or allow the creation of posters to
%% communicate the results of a project.
%% 
%% Use the format Math nnn: Class Title.
%\class{Math 197: Senior Thesis}

%% Advisor(s) name or names.  Separate with \and.
\advisor{Weiqing Gu}

%% Reader(s) name or names.  Separate with \and.
%\reader{Melissa O'Neill \and Charlie Watts}


%% Define the \BibTeX command, used in our example document.
\providecommand{\bibtex}{{\rmfamily B\kern-.05em%
    \textsc{i\kern-.025em b}\kern-.08em%
    T\kern-.1667em\lower.7ex\hbox{E}\kern-.125emX}}


\pagestyle{fancy}

\begin{document}

\begin{poster}

\section{Background}
Scheduling is very important for maintaining order, for both individuals and larger organizations.
In a fast-paced environment like Harvey Mudd, creating schedules can help many people stay organized and manage stress.
With this project, we wanted to develop algorithms that could help create effective schedules capable of adapting do an individual's preferences or style of working.

\section{Algorithm for Personal Scheduling}%

For an individual, such as a Harvey Mudd student, it can be important to create a schedule in order to figure out use their time effectively. 

Our algorithm for creating a personal schedule aims to both accomplish all necessary tasks and maintain a balanced lifestyle.
We treat a schedule as a collection of relatively short \emph{time slots}, and contiguous time slots can form \emph{blocks}.
A group of consecutive blocks can form a \emph{section}, identified by it starting block and ending block.
%TODO Explain what a task is (time slots, due dates, priority)
The algorithm has three main components.
\begin{enumerate}
	\item
		Distribute the tasks into sections of the schedule (on a scale larger than blocks), where all tasks within a section will be completed by the end of section.
	\item
		Assign tasks in each section to a block. 
	\item
		Schedule tasks within each block.
\end{enumerate}






\section{For Further Information}
For more information about this project, please contact \url{sbost@hmc.edu} or \url{gu@hmc.edu}.


\vfill
\columnbreak


\section{Algorithm for Event Scheduling}

\section{Results}%

\section{References}
\bibliographystyle{hmcmath}
%\bibliography{sampleposter}
\vfill

\section{Acknowledgments}
I am very grateful to the Rose	Hills	Foundation Science \& Engineering Summer Undergraduate Research Fellowship program for their kind and generous support of this project.


\vfill
\end{poster}

\end{document}

 
